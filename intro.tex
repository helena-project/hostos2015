\section{Introduction}

As networking became an integral part of computing through the Internet and
world wide web, system security grew in importance from a niche interest
of the military into a core requirement for every operating system and 
software application. Twenty years ago, we had to be careful about what
floppy disks we inserted into our computers. Today, we have firewalls,
NATs, automatic virus scanners, no-execute bits, virtual machines, 
increasing use of secure HTTP, and and a growing infrastructure for 
authenticating downloaded data and code.

Today, we see the emergence of ``intelligent'' things that integrate
computing with physical devices. Smart watches such as the Pebble
provide a clock application and a limited set of applets, smart light
bulbs such as the Phillips Hue support single functions---on and off,
brighter and dimmer---, while smarter light bulbs such as the PLAYBULB
pulse with the music they play from their integrated speaker.  As
embedded systems grow from narrow applications into a more general
purpose computing infrastructure, there will be a need to dynamically
load applications and share the infrastructure across multiple
applications.  This trend is already starting: the Pebble watch, for
example, supports loading executable code that can use the builtin
sensors and display as well as communicate with apps on a connected
mobile phone.

Historically, embedded systems have been single-purpose,
single-application devices.  A developer compiles a monolithic program
image combining OS and application that has total access to
underlying hardware.  This lack of a security model is due to the
limitations of the hardware itself. Microcontroller families such as
the MSP430 and Atmega lack virtual memory and memory protection.  As a
result, embedded ``operating systems'' have no security mechanisms and
applications must be trusted completely.

But in the past few years, a new generation of embedded processors
has emerged. These processors, based on the ARM Cortex M architecture,
sit in a middle ground between full-fledged CPUs and microcontrollers.
Like some mid-1980s processors, they are 32 bits, have memory protection, and
run at tends of MHz. But like microcontrollers, they have tens to
a few hundred of kilobytes of RAM, lack virtual memory,
provide the many peripherals
and bus interconnects (SPI, I2C, one-wire, UART/USART, etc.) that embedded
systems need and 
have sub-\uA sleep currents necessary for battery-based operation.
The Nordic Semiconductor nRF51822, for example,
is an integrated system-on-a-chip (SoC) with a Bluetooth Low Energy 
transceiver and a Cortex M0 processor (16MHz, 32kB RAM, 256kB flash)
costs a little more than \$1. 

While these new embedded processors have slighly more RAM and run slightly
faster than microcontrollers, we argue that their inclusion of memory 
protection is far more significant. Protection is the key abstraction that 
current embedded operating
systems lack but which is essential to enable the next generation
of embedded multi-application devices. Memory protection can isolate 
applications from one another and the kernel from applications. However,
this isolation and safety is only as good as the kernel itself. As numerous
recent results have shown, OS kernels contain scores of memory access
bugs that 
applications can use to gain unauthorized access~\cite{nickolai}. We can
take protection one step further by writing the OS kernel in a type-safe
language.  The
relative simplicity of Cortex M processors and embedded operating system
APIs means that writing such a kernel is a small and manageable task,
still open to easy extension and evolution.

This paper argues that it is time for embedded operating systems to graduate 
from  a glorified 
hardware abstraction layer to a real operating system, 
one that enables and encourages multiple concurrent, isolated applications,
provides both hardware abstractions and low-level hardware access, and offers
critical services for the applications of the Internet of Things.
As a concrete example of what such an operating system might look like,
it presents \name, a new operating system that combines hardware-based
memory protection and language-based type safety to provide a secure,
robust platform for embedded applications. The need to integrate untrusted
application code in C with a type-safe kernel raises interesting challenges
in how data passes between them and how the OS manages its memory layout.
