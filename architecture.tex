\section{\name: a Secure Embedded OS}

In our ongoing project, \name, we are building a new embedded operating system
specifically designed for platforms running multiple, potentially untrusted
applications concurrently and third-party, contributed device drivers that are
assumed to be buggy. \name uses three protection mechanisms to protect three
different components in the operating system. The kernel itself is written in
Rust, allowing \name to use language level protection, such as memory safety and
ownership, to protect the core of the kernel (e.g. the scheduler and hardware
abstraction layer) from contributed device drivers. \name uses software managed
memory protection to isolate third-party applications from each other and the
kernel. Finally, where available, multiple microcontrollers on the same platform
are use to protect applications with timing constraints against starvation and
in cases where timing and other side-channels are a concern. The remainder of
this section gives an overview of the \name architecture and dives into \name's
use of the memory protection unit and application memory layout.

\subsection{Architecture}

In traditional embedded operating systems, a single application is
compiled along with a library operating system to a single executable that is
loaded on to the target device. The application, device drivers, scheduler, and
other OS services run with the same CPU privileges, share memory and have equal access
to the same hardware resources. \name takes a radically different approach. In
\name, there is a standalone, portable kernel that is composed, at compile-time,
with approriate device drivers for the platform and enables dynamic loading of
third-party applications at runtime.

At a high level, \name has three layers of computational units. At the lowest
layer, the core of the kernel---the scheduler, task manager, hardware
abstraction layer, etc---has completely control of the hardware, including
memory-unsafe and thread-unsafe operations such as writing to arbitrary memory
and turning interrupts on and off. A device driver runs with the same
\emph{hardware} privileges as the core kernel but inside a compile time,
\emph{language-level} sandbox. The language sandbox enforces that device drivers
can only use the kernel provided hardware abstractions to access hardware, that
dynamic memory allocation is limited to load time and stack allocation, and that
drivers cannot interfere with each other by accidentally sharing underlying
hardware resources. For example, the Rust language has a feature called
``ownership`` (which originally appeared in the Cyclone language~\cite{cyclone})
that ensures, at compile time, there is only one active reference to a
value not marked copyable. In \name, we use this feature to give device drivers
``ownership'' over hardware resources and ensure those resources are not a
vector for race conditions between drivers. Moreover, \name relies on Rust's
strong module boundries and type system to restrict device drivers from
accessing unsafe internals of the hardware abstraction layer.

The third layer, where applications run, is isolated using hardware protection
mechanisms. Applications, which do not have to be written in Rust, run in an
unprivileged CPU mode that restricts access to privileged CPU and hardware
registers, such as the control and program status CPU registers and the memory
protection unit hardware registers. Applications can access hardware resources
in two ways. First, through a system call interface exposed by the kernel and
device drivers.  Second, the kernel can give an applications direct read and/or
write access to particular hardware registers.

In cases where the platform has multiple microporcessors, \name can use
additional CPUs to isolate applications at yet another layer. We envision that
in multi-application embedded platforms, most applications will not be very
sensitive to reasonable scheduling decisions, however some applications will
undoubtedbly have timing constraints. \name will leverage multiple
microprocessors by scheduling applications on the microprocessor with the most
relevant hardware---for example, a communication bound application will be
scheduled on a Bluetooth SoC. Moreover, some microprocessors on a platform will
run instances of \name with a real-time scheduler. Finally, we plan to leverage
physically separate microprocessors to protect applications dealing with
particularly sensitive data (such as encryption keys, or end-user medical
information) from side-channel attacks by running them on a separate
processor~\cite{trustzone}.

% \subsection{The Cortex-M MPU}
% 
% The Cortex-M MPU allows the kernel to defined eight memory regions. Each memory
% region may be sized between 32 bytes and 4GB, in 32 byte increments and has
% protection bits for read, write and execute. Regions may overlap, in which case
% the memory region with the highest number ``wins''. In addition, regions of at
% least 256 bytes can be divided into eight equal sized subregions, which can
% either be turned on---in which case they inheret the parent region's protection
% bits---or turned off---in which case the parent's protection bits do not apply.
% As a result, the operating system can control access to up to 64 concurrently
% active regions. Finally, memory regions and protection bits can be changed
% during exection, for example while context switching between applications.

\subsection{OS Design Challenges}

To design an operating system for today's embedded hardware running multiple
untrusted applications concurrently we must rethink some of the core mechanisms
used in modern operatingy systems for other environments (specifically, the
desktop, mobile and server). For example, the main mechanism for isolating
applications from each other and the kernel in modern desktop operating systems
is virtual memory or segmentation. These mechanisms are inapproriate in an
embedded system in several ways, not least of which that hardware support for
such mechanisms is simply not available.

On the other hand, new microporcessors do support memory protection, which
unlike virtual memory, exposes a flat address space to applications and the
kernel, but allows the kernel to set access control rules on memory regions as
small as 32 bytes. \name explores one point of the design space of how to best
layout kernel and application memory, design a system call interface and mediate
access to hardware using an MPU.

Figure~\ref{fig:memory-layout} shows the four types of memory regions in \name
as well as the access rules when an application is running for those regions and
subregions.

\begin{enumerate}
  \item The {\bf code} region contains kernel, device driver and application
    code. While an application is executing code, is has read and instruction
    fetch access only to it's own code segment. It cannot read code for other
    applications or the kernel since a compiler might inline sensitive constants
    which could then appear in the code section. An application cannot write to
    it's own code section since, in most cases, the code is backed by internal
    flash, which has limited write cycles.

  \item {\bf Hardware memory registers} are typically located in a single,
    continuous region of memory.~\endnote{the location of hardware memory
      registers and flash is determined by the chip manufacturer, however, chips
      chips typically place flash at the bottom of memory, followed by RAM, and
    peripheral memory registers towards the top of memory.} \name may provide
    read-only, or read-write access over small ranges of memory registers to
    applications. For example, our development platform uses the Atmel SAM4L
    Cortex-M4 which provides a true random number generator peripheral. The
    relevant registers for reading a new random value are contained in an 20
    byte range.  \name exposes random numbers directly to applications by allow
    read-only access to a 32 byte containing the relevant registers (the tailing
    12 bytes are innocuous).

  \item The {\bf Kernel's stack and static data} are located at the top of RAM.
    They include all kernel and device driver buffers and any local variables so
    applications may not access the kernel stack in any way. The kernel does not
    allocate memory dynamically except on the stack and inside application
    memory regions, as discussed below. As a result, the kernel stack is the
    only region dedicated to kernel memory.

  \item \name allocates {\bf application memory} in the remaining space below
    the kernel stack. Currently, \name allocates a continuous, fixed size region
    for each application (our development platform has 64KB of RAM and we
    currently use 2KB memory regions for applications, but this will likely vary
    based on the requirements of each target platform). Application memory and
    the kernel stack grow towards each other to provide the most flexibility in
    number of applications vs. kernel stack size. If the kernel stack grows too
    large, \name always prefers to terminate the application with the largest
    memory region. An alternative memory allocation technique would be to
    allocate application memory as requested from a shared heap (which is
    possible due to the high granularity of the memory protection unit). We have
    not yet explored the tradeoffs between these two strategies, but at a high
    level, but fixed sized application memory regions makes it easier for the
    kernel to reclaim stack space under memory pressure while a shared heap
    provides heterogenously sized application with more flexibility.

\end{enumerate}

An application's memory region is not entirely it's own. Most of it's memory
used for it's stack and static variables (at the bottom), and heap. However, the
kernel may also ``borrow'' memory from the top of an application's memory region
for application-specific kernel data structures. This region, which may change
in size, is \emph{inside} the application's memory space, but is marked as read
and write protected when the application is running. Therefore, the kernel, or device
drivers, can allocate memory dynamically in response to application without
sacrificing the reliability of never dynamically allocating in the kernel itself
and while mainting integrety and confidentiality of shared kernel data
structures. For example, when an application registers to a timer callback, the
timer device driver adds the request to it's list of pending timers by
allocating the new list node to in the application's memory. Since the link includes
forward and backward list pointers, it's imperative that despite being in the
application's memory, this link's integrity be maintained. Moreover, the values
of the links might leak information about other applications.

