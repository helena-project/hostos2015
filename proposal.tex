\documentclass{article}
\title{Microcontrollers Deserve Protection Too}
\author{}
\date{}

\begin{document}
  \maketitle

  \section{Overview}

  Traditionally, microcontrollers ran monolithic applications that were written
  once to run for months or years untouched~\footnote{In a forest somewhere in
  the Amazon, collecting data about honey bee mating behavior.}. As a result,
  unlike desktop and server operating systems, embedded operating systems like
  TinyOS assume that applications are trustworthy and give them unfettered
  access to the underlying hardware. However, we believe the world has changed.
  Modern microcontrollers power consumer products like fitness bands and smart
  watches~\footnote{the Pebble smartwatch and Fitbit both run on a STM
  Cortex-M3, while the Jawbone UP runs on an MSP420}. These products often
  include multiple independent applications sharing the hardware which may be
  installed and/or updated by the end-user. We must rethink how we design
  embedded operating systems to accomodate these new trends, understanding that
  traditional desktop operating systems are too power hungry and rely on
  hardware mechanisms (such as an MMU) not available in embedded
  microporcessors.

  Tock is a next-generation embedded operating system for the ARM Cortex-M
  series of microprocessors. Tock provides isolation and protection between
  applications and the kernel using three mechanisms at different levels:

  \begin{enumerate}
    \item \emph{Language level protection} Tock is written in Rust, a type-safe,
    compiled, low-level language. Tock's enforces isolation between it's
    internal components (e.g. core device drivers) using language-level
    protection~\footnote{Yeah, I know, this is really vague right now}.
    
    \item \emph{Software-managed memory protection} Modern ARM Cortex-M
    processors provide a hardware protection mechanism called Memory Protection
    Unit (MPU).  An MPU allows the kernel to set access permissions on a fixed
    number of memory regions which are enforced on application code. Unlike the
    Memory Manage Units (MMUs) in application processors, MPUs do not provide
    virtual addressing, however, like MMUs illegal accesses to memory regions
    (e.g. writing to read-only memory) are caught and result in a fault to the
    kernel. Tock uses the MPU to protect kernel and driver memory from untrusted
    application code and different applications from one another, while
    providing applications direct access to hardware registers, such as the
    GPIO.

    \item \emph{Physically isolated cores} Modern embedded platforms often
    involve multiple microprocessors in practice. For example, both the CC2540
    and the NRF51822 Bluetooth modules are, in fact, full blown microprocessors,
    but many products include them in addition to another
    microprocessor~\footnote{Well, at least Coin does that}. Tock leverage
    multiprocessor environments to enforce another layer of isolation between
    applications and the kernel, by running different application components on
    different cores depending on their needs to access specific hardware, as
    well as power and performance constraints.

  \end{enumerate}

  \section{Challenges}

  \begin{enumerate}
    \item The compiler and OS should synthesize a multi-core execution plan
    subject to developer configuration/annotation in the code and, potentially,
    real time constraints.

    \item What is the execution model?

    \item The kernel must provide a way of inspecting the state of the system
    during debugging. For example, in TinyOS, timer callbacks are determined
    statically, but this would not be the case in a syscall based system with
    dynamic applications. The kernel should therefore provide a way to query,
    e.g., which timer's are currently outstanding.

  \end{enumerate}
\end{document}
