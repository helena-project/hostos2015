\section{The Next-Generation Embedded OS:\\Protection \emph{is} a New Primitive}
\label{protection}

% pp 33--34, ch 1, sec 4:
In Tanenbaum's \emph{Modern Operating Systems} he asserts, ``The main property
which distinguishes embedded systems from handhelds is the certainty that no
untrusted software will ever run on it''~\cite{tanenbaum}. In this paper we
argue that this is no longer the case.
% As a class of device, embedded systems
% are evolving beyond the single-application case, and as a consequence, they
% require their operating system to provide process isolation, enforced resource
% arbitration, and protection.
Embedded systems
are evolving beyond the single-application case and now
require an operating system that provides process isolation, enforced resource
arbitration, and protection.

% Today, we are designing \name, a new embedded operating system that (amongst
% other goals) explores how to add protection as a primitive to constrained
% systems
\name explores how to add protection as a primitive to constrained
systems.
We begin by examining recent advances and trends in embedded
hardware and systems language design with an eye towards applying these
new capabilities to our embedded OS concept.

%\name takes advantage of three emerging trends, modern language design, new
%microcontroller hardware capabilities, and contemporary multi-MCU system design,
%that afford new opportunites to provide protection.

%\subsection{Memory Protection}

Modern ARM Cortex-M processors provide a hardware protection mechanism called
Memory Protection Unit (MPU).  An MPU allows the kernel to set access
permissions on a fixed number of memory regions which are enforced on
application code. Like Memory Management Units (MMUs) found in application
processors, MPUs trap illegal memory accesses (e.g. writing to read-only
memory) to the kernel, however unlike the MMUs, they do not provide virtual
addressing, so do not enable mechanisms like swap memory, shared libraries,
etc. MPUs are typically much more fine-grained than virtual memory. For
example, in the ARM Cortex-M series of microcontrollers, the MPU can address a
mix of region sizes as small as 32 bytes and as large as 4GB, whereas virtual
memory typically divides memory into fixed sized pages of at least 4KB.

\name uses the MPU to protect kernel and driver memory from untrusted
application code as well as different applications from one another.
Applications are given dedicated region of memory for stack, heap and appliction
specific kernel buffers. When execution is yielded to an application, the MPU
restricts access to memory outside of this region. Moreover, application
specific kenel buffers (e.g. interrupt callback queues) are allocated in the
application's memory region, but protected from application tampering. This is
possible due to the fine granularity of the MPU and allows \name to eliminate
dynamic allocation in the kernel but allow more runtime flexibility in
applications.

%\subsection{Multiple Processors}

Modern embedded platforms increasingly include multiple microprocessors. Often,
peripheral devices like radio modules are actually a full blown microprocessor
that to allow the application controller to offload low-level radio
functionality and reduce overall power~\cite{nrf51822,cc2540}. In other cases, a
similar microprocessor with a different energy profile to provide applications
with more control over the energy-performance tradeoff.  For example, the Nest
Protect includes a main Cortex-M4 application controller, a secondary Cortex-M0+
peripheral processor and the EM357 802.15.4 radio SoC, which has an additional
onboard Cortex-M3~\cite{nestprotect-teardown}.

\name leverages multiprocessor environments to provide protection where
language-level and memory isolation are insufficient. In a multi-application
environment, time-sensitive applications (e.g. that must process requests
within a single radio scheduling quantum) are protected from interference by,
e.g compute heavy applications. In addition, \name schedules applications on
different processors to applications handling sensitive data from side-channel
attacks by other applications.



\subsection{Hardware Advances and Trends}

New embedded system platforms will be built on the Cortex-M series of microcontrollers.
While ARM's A-series is substantially more capable and is used in wall-powered
devices such as the Raspberry Pi~\cite{rpi} and BeagleBone Black~\cite{bbb},
its significant power draw (over 1~W in active mode) make it infeasible for
low-power, battery operated devices.
In contrast, the Cortex-M series microcontrollers have power draws conducive
to low-power operation while adding hardware features that enable protection.


% Modern ARM devices come in three flavors: the powerful A-series
% microprocessors, the real-time R-series, and the efficient M-series
% microcontrollers.
% \hl{XXX: Why M over A}

{\bf Memory Protection Units.}
In lieu of costly Memory Management Units (MMUs), the Cortex-M series includes
a Memory Protection Unit (MPU). Like MMUs, MPUs trap illegal memory accesses
(e.g. writing to read-only memory).
With OS assistance an MPU
with position independent code (PIC) can enable isolated applications, shared
libraries, and even possible relocation of running code.
% Pat: I'm 90% sure of how to do relocation with PIC, but I need to think about it more
Further, MPUs are typically much more fine-grained than
virtual memory and are able to address regions as small a 32~bytes, whereas MMUs
typically use at least 4~KB pages.
MPUs do not, however, perform any translation, so mechanisms such as swap are unavailable.

{\bf ``Multi-Core'' MCUs.}
% XXX: This section is a work-in-progress...
Modern embedded platforms increasingly include multiple microcontrollers.
For example, the Nest
Protect~\cite{nestprotect} includes a main Cortex-M4 application controller, a secondary
Cortex-M0+ peripheral processor and an EM357 802.15.4 radio SoC with an
onboard Cortex-M3~\cite{nestprotect-teardown}.
Adding additional microcontrollers can allow for offloading timing-critical
communication functionality or running certain computations on a microcontroller
with a different energy profile.
% Often, peripheral devices like radio modules are actually a full blown
% microcontroller that allows the application controller to offload low-level radio
% functionality and reduce overall power~\cite{nrf51822,cc2540}. In other cases,
% a similar microcontroller with a different energy profile is added to provide
% a runtime option for an energy-performance tradeoff.
Multiple
%, hetergeneous
processors provide an opportunity for
the strictest isolation, stochastic and deterministic real-time schedules, and
computation offloading. It is less clear, however, how to abstract
system-specific hardware for general applications and is an open question in
the design of \name.


% ARM TrustZone is also functionally like having a separate core, or at least
% that's the abstraction it tries to provide. Only available on Cortex-A's
% though

% Could also be interesting to talk about crypto co-processors here



%% OLD TEXT: impl-heavy
% \name leverages multiprocessor environments to provide protection where
% language-level and memory isolation are insufficient. In a multi-application
% environment, time-sensitive applications (e.g. that must process requests
% within a single radio scheduling quantum) are protected from interference by,
% e.g compute heavy applications. In addition, \name schedules applications on
% different processors to applications handling sensitive data from side-channel
% attacks by other applications.

%Advances:
%  MPUs
%    -- capabilities
%    -- MPU vs MMU (seed of both?)
%
%Trends:
%  M vs A
%    -- Why deeply embedded (M) still relevant
%    -- Maybe Pebble vs other watches example?
%
%  Multi-Core (really Multi-MCU?)
%    -- solid protection / boundary
%    -- compare with TrustZone?
%    -- App / HW specific; need to expose but can't feature / focus / rely on


\subsection{Language Level Protection}

The C programming language has long been the language of choice for system-level
programming, including operating systems development. C is inherently an unsafe
programming language \cite{kint:osdi2012, undefined:apsys2012}. As such, much
effort has been made to develop operating systems in high-level languages with
stricter semantics \cite{singularity:sigops, house:icfp2005, unikernels:2013}.
Type safety, memory safety, and strict aliasing each provide support for
operating system protection at the language level.

\paragraph{Strong Type Safety.}
Strong, strict types allow for abstractions of low-level hardware mechanisms
that cannot be subverted by safe code. Because the type system cannot be
subverted, hardware mechanisms correctly modeled as type values exposing typed interfaces
guarantee that users of the interface are utilizing the underlying hardware in a
safe manner.
% Of course, the underlying interface implementation continues to be
% responsible for implementing the desired logic correctly.
Taking advantage of strict typing requires minimizing the amount of unsafe code,
i.e. code that interfaces with hardware directly.
Although challenging, we argue that careful interface design will allow
for minimum unsafe code in the operating system and the maximum cost-free
protection from a strong type system.
% As such, it is
% important to design interfaces that minimize the ability for logic errors to be
% introduced via their implementation. We argue that through careful interface
% design, unsafe code, or code that interfaces with the hardware directly, can be
% minimized, reducing the amount of code that must be audited to ensure
% correctness. Identifying the minimum amount of unsafe code to expose a safe
% interface is a core challenge, but successfully doing so results in strong,
% cost-free protection.

\paragraph{Memory Safety.}
Issues with memory safety---dangling pointers, use-after-free and
double-free errors, access to unallocated memory, and pointer arithmetic
errors---have long plagued operating systems written in
languages with weak memory semantics.
% Dangling pointers, use-after-free and
% double-free errors, access to unallocated memory, and pointer arithmetic errors
% are a few of the issues kernel developers encounter when writing in such a
% language.
A language that guarantees memory safety ensures that some or all of
these types of errors cannot occur once a program has compiled.
For instance, a kernel with drivers written in a memory safe language are
memory isolated without the need for hardware support as the
application can only access memory that has been allocated to it.
% A memory safe application can only access memory that has been allocated to it.
% This means that a kernel with drivers written in a memory safe language are
% memory isolated without the need for hardware support: It is impossible for a
% memory-safe driver to sully the integrity of a kernel written in the same
% language.

% A high-level language usually guarantees memory safety through automatic garbage
% collection and bounds checking; the language allocates and frees all memory for
% the user and checks all pointer arithmetic. Because the garbage collector tracks
% active references, automatic garbage collection imposes a runtime performance
% penalty that is difficult to determine deterministically. In a constrained
% system, where applications may require tight bounds on their execution,
% deterministic performance is critical. As such, automatic garbage collection is
% unlikely to be a good fit for embedded systems.

The kernel, however, must still be able to access memory directly
to configure hardware resources.
% Of course, to program hardware, the kernel must be able to access memory
% directly.
As with type safe interfaces, unsafe accesses must be audited to ensure
the overall interface is memory safe, enabling both safe drivers and
% the ideal is to provide a memory safe
% interface to carefully audited memory unsafe operations. This guarantees the
% memory safety of software utilizing the memory safe interfaces, such as drivers,
% while enabling
the kernel's low-level implementation.

\paragraph{Strict Aliasing.}
Strict aliasing rules, such as unique references and read/write references,
%are used to
enforce thread safety
%in modern programming languages
by the language. We seek to
exploit these semantics
% to provide thread-safe hardware access
by modeling
hardware resources as references, allowing two or more applications to run
concurrently without
%fear of race conditions when accessing hardware
hardware race conditions.
% This means that two or more concurrently running applications may
% attempt to access the same hardware, resulting in access patterns unanticipated
% by the hardware. We seek to exploit strict aliasing semantics, such as unique
% and read/write references, by modeling hardware resources as references,
% guaranteeing thread-safe hardware access.
\textit{Unique references}
to hardware
guarantee that only a single active execution context
can access the reference,
%By modeling hardware as a unique reference, a
guaranteeing thread safety to applications holding the reference.
\textit{Read/write references} guarantee that two active contexts can read, but cannot mutate,
the same state.
% If two or more applications require read access to the same
% resource, read/write references can be used to make the same guarantee.
If multiple applications require write access to the same resource, a broker with a
unique reference to the resource can mediate access.
%to said resource.

By composing type safety, memory safety, and strict aliasing, a next-generation
operating system can expose safe interfaces to applications and device drivers
with strong protection guarantees at little performance cost.


