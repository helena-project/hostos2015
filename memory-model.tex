\section{Memory Model}

Unlike desktop operating systems, microcontrollers with memory protection have
no support for virtual addressing. As a result, any executing thread can attempt
load and store operations to any address in memory, including sensitive kernel
buffers or hardware memory registers. The memory protection unit (MPU) allows an
operating system to protect sensitive memory regions from malicious or buggy
applications by marking memory regions with read, write and execute permission.
These permission take effect when the processor is in unprivileged mode.
However, using access control instead of virtual memory for protection requires
well behaved applications to know exactly where memory regions reside and which
regions it has access to.

\subsection{The Cortex-M MPU}

The Cortex-M MPU allows the kernel to defined eight memory regions. Each memory
region may be sized between 32 bytes and 4GB, in 32 byte increments and has
protection bits for read, write and execute. Regions may overlap, in which case
the memory region with the highest number ``wins''. In addition, regions of at
least 256 bytes can be divided into eight equal sized subregions, which can
either be turned on---in which case they inheret the parent region's protection
bits---or turned off---in which case the parent's protection bits do not apply.
As a result, the operating system can control access to up to 64 concurrently
active regions. Finally, memory regions and protection bits can be changed
during exection, for example while context switching between applications.

\subsection{Memory Regions in \name}

The operating system has roles with respect to application memory. First it
manages where applications are allowed to allocate memory. Second it manages
which memories the application can access. This includes most of the application
allocated memory, but may also include hardware memory registers that are
exposed directly to applications.

{\bf Allocation Regions.} 
In \name, the kernel allocates three separate memory regions for each
application, one for each of the stack, heap and data sections. The stack and
data sections are completely accessible to the application, and only the
application can allocate memory there. On the other hand, the heap is shared
between the application and the kernel, which \emph{borrows} heap memory from
the application for things like buffers and queues which the kernel needs
to satisfy application demands.

{\bf Access Rules.}
