Microcontroller operating systems and frameworks typically assume that a single,
monolithic application will run on an embedded system. In fact, historically,
power consumption, memory and code sizes, and lack of hardware support have
constrained microcontroller applications to be single-function. This is no
longer the case with modern hardware. Modern microcontrollers, like the Atmel
SAM4L (an ARM Cortex-M4), provides over six times more SRAM and over five times
larger flash than the TI~MSP430 that powered the TelosB motes while maintaining
similar power draws (90~{\uA}/MHz active and 3~\uA at sleep).
Simultaneously embedded products are becoming a development platforms and an
application ecosystems for products like the Pebble watch. Additionally, a set
of modular embedded devices is emerging. Devices like SimBand, Wzzard and
ThinkingThings have a core part to which additional modules can be attached to
extend the functionality. These modules can be developed by the third parties
and have own microcontroller with runtime environment.
The operating systems community should leverage advancements in hardware,
programming languages as well as our experience from the Web, and other
application rich ecosystems, to build the next generation of embedded operating
systems.

