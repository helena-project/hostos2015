\section{Introduction}

Traditionally, microcontrollers ran monolithic applications that were written
once to run for months or years untouched~\endnote{In a forest somewhere in
the Amazon, collecting data about honey bee mating behavior.}. As a result,
unlike desktop and server operating systems, embedded operating systems like
TinyOS assume that applications are trustworthy and give them unfettered
access to the underlying hardware. However, we believe the world has changed.
Modern microcontrollers power consumer products like fitness bands and smart
watches~\endnote{the Pebble smartwatch and Fitbit both run on a STM
Cortex-M3, while the Jawbone UP runs on an MSP420}. These products often
include multiple independent applications sharing the hardware which may be
installed and/or updated by the end-user. We must rethink how we design
embedded operating systems to accomodate these new trends, understanding that
traditional desktop operating systems are too power hungry and rely on
hardware mechanisms (such as an MMU) not available in embedded
microporcessors.

The key abstraction lacking in traditional embedded operating systems, but
essential to enable the next generation of embedded devices is protection. In
operating systems like TinyOS and Contiki, applications are linked into the
operating system at compile time and all parts of the code have access to the
same address- and name-space. FreeRTOS has a notion of \emph{tasks} which run
independently of the operating system, but this separation is not enforced --
applications have complete access to operating system memory regions and
unfettered hardware access.

We argue that embedded operating systems should take advantage of three
protection mechanisms recently available to embedded systems builders:

\begin{enumerate}
  \item \emph{Language level protection} including type safety, memory safety
    and thread safety support in modern systems programming languages.
  \item \emph{Software-managed memory protection} available in the newest
    low-power embedded microcontrollers allow the operating system to enforce
    privilege separation in hardware without the cost or complexity of a virtual
    memory.
  \item \emph{Physically timing protection}. Many new embedded products
    integrate multiple microcontrollers~\endnote{Nest protect has no less than
    three Cortex-M microcontrollers. An M4 for the application controller, an
  M0+ used as a peripheral coprocessor and 802.15.4 SoC with an onboard M3}.
  Operating systems should take advantage of this to guard time-sensitive
  components from others that might compromise timing guarantees (e.g.
  computation heavy tasks).
\end{enumerate}

We are exploring these protection mechanisms in a new embedded operating system
called \name. We are designing \name specifically for embedded products that
will allow end-users to install third-party applications during deployment and
that incorporate arbitrary new devices in many configurations.

\subsection{Language Level Protection}

\name is written in Rust: a type-safe, thread-safe, compiled, low-level systems
language. Using a language with strong safety protections has enabled \name to
formalize the separation between core kernel code and contributed device
drivers, which are traditionally where most bugs are found but are included in
the kernel for performance and practical reasons. While recent server operating
systems utilize hardware I/O virtualization~\cite{arrakis:osdi2014, ix:osdi2014}
to run device drivers outside the kernel, such hardware mechanisms are unlikely
to be available in microcontrollers in the foreseable future.

\subsection{Memory Protection}

Modern ARM Cortex-M processors provide a hardware protection mechanism called
Memory Protection Unit (MPU).  An MPU allows the kernel to set access
permissions on a fixed number of memory regions which are enforced on
application code. Unlike the Memory Manage Units (MMUs) in application
processors, MPUs do not provide virtual addressing, however, like MMUs illegal
accesses to memory regions (e.g. writing to read-only memory) are caught and
result in a fault to the kernel. Tock uses the MPU to protect kernel and driver
memory from untrusted application code and different applications from one
another, while providing applications direct access to hardware registers, such
as the GPIO.

\subsection{Multiple Processors}

Modern embedded platforms often involve multiple microprocessors in practice.
For example, both the CC2540 and the NRF51822 Bluetooth modules are, in fact,
full blown microprocessors, but many products include them in addition to
another microprocessor~\endnote{Well, at least Coin does that}. Tock leverage
multiprocessor environments to enforce another layer of isolation between
applications and the kernel, by running different application components on
different cores depending on their needs to access specific hardware, as well as
power and performance constraints.

