\subsection{Memory Protection}

Modern ARM Cortex-M processors provide a hardware protection mechanism called
Memory Protection Unit (MPU).  An MPU allows the kernel to set access
permissions on a fixed number of memory regions which are enforced on
application code. Like Memory Management Units (MMUs) found in application
processors, MPUs trap illegal memory accesses (e.g. writing to read-only
memory) to the kernel, however unlike the MMUs, they do not provide virtual
addressing, so do not enable mechanisms like swap memory, shared libraries,
etc. MPUs are typically much more fine-grained than virtual memory. For
example, in the ARM Cortex-M series of microcontrollers, the MPU can address a
mix of region sizes as small as 32 bytes and as large as 4GB, whereas virtual
memory typically divides memory into fixed sized pages of at least 4KB.

\name uses the MPU to protect kernel and driver memory from untrusted
application code as well as different applications from one another.
Applications are given dedicated region of memory for stack, heap and appliction
specific kernel buffers. When execution is yielded to an application, the MPU
restricts access to memory outside of this region. Moreover, application
specific kenel buffers (e.g. interrupt callback queues) are allocated in the
application's memory region, but protected from application tampering. This is
possible due to the fine granularity of the MPU and allows \name to eliminate
dynamic allocation in the kernel but allow more runtime flexibility in
applications.
