\section{Introduction}

As networking became an integral part of computing through the Internet and
world wide web, system security grew in importance from a niche interest of the
military into a core requirement for every operating system and software
application. Twenty years ago, we had to be careful about what floppy disks we
inserted into our computers. Today, we have firewalls, NATs, automatic virus
scanners, no-execute bits, virtual machines, increasing use of secure HTTP, and
a growing infrastructure for authenticating downloaded data and code.

Today, we see the emergence of ``intelligent'' \textit{things} that integrate
computing with physical devices. Smart watches such as the Pebble~\cite{pebble}
provide a clock application and a limited set of applets. Smart light bulbs such
as the Phillips Hue~\cite{hue} support simple functions--- turn on, turn blue,
glow brighter---, while smarter light bulbs such as the PLAYBULB~\cite{playbulb}
pulse to the music playing from their integrated speaker.  As embedded systems
grow from narrow applications into a more general purpose computing
infrastructure, there will be a need to dynamically load applications and share
the infrastructure across multiple applications.  This trend is already
starting: the Pebble watch, for example, supports loading executable code that
can use the built-in sensors and display as well as communicate with apps on a
connected mobile phone.

%on and off, brighter and dimmer---,

But historically, embedded systems have been single-purpose, single-application
devices.  A developer compiles a monolithic program image, combining the OS and
its applications into one bundle that has total access to underlying hardware.
We argue that this lack of a security model is in part due to the limitations of
the hardware itself. Microcontroller families such as the MSP430 and Atmega lack
virtual memory and memory protection.  As a result, embedded ``operating
systems'' have no security mechanisms and applications must be trusted
completely.

In the past few years however, a new generation of embedded processors has
emerged. These processors, based on the ARM Cortex-M architecture, sit in a
middle ground between full-fledged CPUs and microcontrollers. Like some
mid-1980s processors, they have a 32-bit address space, memory protection, and
run at tens of MHz. But like microcontrollers, they have tens to a few hundred
of kilobytes of RAM, lack virtual memory, provide many peripherals and bus
interconnects (SPI, I2C, one-wire, UART/USART, etc.), and have sub-\uA sleep
currents necessary for battery-based operation. The Nordic Semiconductor
nRF51822~\cite{nrf51822}, for example, is an integrated system-on-a-chip (SoC)
with a Bluetooth Low Energy transceiver and a Cortex-M0 processor (16MHz, 32kB
RAM, 256kB flash) and costs a little more than \$1.

The inclusion of memory protection is particularly significant: protection is
the key abstraction absent in current embedded operating systems that will
enable the next generation of embedded multi-application devices. Memory
protection can isolate applications from one another and the kernel from
applications. However, this isolation and safety is only as good as the kernel
itself. As numerous recent results have shown, OS kernels contain scores of
memory access bugs that applications can use to gain unauthorized
access~\cite{nickolai}. By writing the OS kernel in a type-safe, memory-safe
language with strong semantics, we can curtail these types of vulnerabilities.
The relative simplicity of Cortex-M processors and embedded operating system
APIs means that writing such a kernel is a manageable task.
% still open to easy extension and evolution.

We believe it is time for embedded operating systems to graduate from  a
glorified hardware abstraction layer to a real operating system, one that
enables and encourages multiple concurrent, isolated applications, provides both
hardware abstractions and low-level hardware access, and offers critical
services to the applications of the Internet of Things. As a concrete example
of what such an operating system might look like, we present \name, a new
operating system that combines hardware-based memory protection and
language-based safety to provide a secure, robust platform for embedded
applications.

% Sergio's note: This isn't particularly difficult since we've already discussed
% that the kernel and applications will be isolated by hardware. The language
% the applications are written in is irrelevant when this is the case.
%
% The principal challenge that arises is safely and correctly
% integrating untrusted application code in C with a type-safe kernel,
% particularly in how data passes between them and how the OS manages its memory
% layout.
