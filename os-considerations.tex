\section{Operating System Considerations}

We believe that new trends in the ecosystem of embedded applications require a
new set of goals for embedded opearting systems. \name is a next-generation
embedded operating system we are currently developing that takes advantage of
recent developments in microcontroller harware and modern programming language
features to enable multi-application embedded products. \name is designed to
support a number of goals:

{\bf Event-driven execution model}.

{\bf Flexible application programming}. Unlike previous embedded operating
systems like TinyOS~\cite{tinyos} and Contiki~\cite{contiki} which are linked
directly into the application, \name provides a system-call interface. This
allows developers to write applications in any language that can support the
ABI. For example, the \name kernel is written in Rust~\cite{rust}, applications
can trivially written in C, and we are working on a port of Lua~\cite{lua}.

{\bf Flexible kernel}. Embedded hardware platforms are extremely varried. While
\name specifically targets the ARM Cortex-M series of microcontrollers, platform
developers are free to combine a microcontroller with any number of peripheral
devices connected over busses or directly through the GPIO. This means platform
developers must be able to customize the device drivers compiled with the kernel
to their platform. \name follows previous operating systems in separating
drivers for core peripherals (SPI, USART, GPIO, etc) and device drivers (radio,
flash, etc) into separate layers. \name differs from previous operating systems
in enforcing safety policies on device drivers through the Rust type system.
\name prevents drivers from subverting Rust's memory satefy by restricting
device drivers to a safe subset of the Rust language.~\endnote{Rust allows code
to circumvent the type system using the \tt{unsafe} keyword. \name uses a
compiler flag that disallows this keyword when compiling device drivers.} \name
also ensures, at compile time, that at most one driver has access to a specific
hardware resource---multiplexing must be done explicitly in the core peripheral
driver or through an intermediate interface. Finally, \name ensures device
drivers cannot corrupt kernel memory through careful choice of interfaces. \name
ensures that \name does not protect the kernel from denial of service attacks by
drivers.

{\bf Reliability}. Unlike most desktop and server applications, embedded
applications must continue to run without end-user intervention. There is no
console to indicate to the user that an application has crashed. Even if a crash
could be communicated to the user, there is little action she could take. While
a Blue-Screen-Of-Death is annoying on a desktop or server, it is unacceptable in
embedded systems. Unlike other embedded operating systems, in \name,
applications cannot corrupt the kernel or other applications. Moreover, certain
parts of the kernel (e.g. contributed device drivers) are isolated using
a strong language type system.

{\bf Direct Access to Hardware}. Embedded software developers are accustomed to
programming with direct access to hardware. 

{\bf Power Consumption SleepWalking}
