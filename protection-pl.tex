\subsection{Language Level Protection}

Embedded operating systems, like their desktop and server counterparts, have
traditionally been developed using C, or C-derived languages. These languages
lack many protection features of more modern systems languages---memory and
thread safety, a strongly enforced module system---and are inapproriate for
sandboxing untrusted linked code. This has meant that, while device drivers are
often contributed by external developers, likely receive less attention than the
core kernel code and are most likely to contain security vulnerabilities, they
run with the same privileges as core kernel modules, like the scheduler. This
problem is particularly salient in embedded systems frameworks like Arduino,
where platforms are often constructed by cobbling various device ``shields''
together and downloading or copy-pasting device driver code. While recent server
operating systems utilize hardware I/O virtualization~\cite{arrakis:osdi2014,
ix:osdi2014} to run device drivers outside the kernel, such hardware mechanisms
are unlikely to be available in microcontrollers in the foreseeable future.

\name is written in Rust: a type-safe, thread-safe, compiled, low-level systems
language. Using a language with strong safety protections enables \name to
enforce the separation between core kernel code and contributed device
drivers, which are traditionally where most bugs are found but are included in
the kernel for performance and practical reasons. Moreover, resource sharing
between drivers (e.g. of communication busses) must be done {\bf explicitly}, and
by default, the type system ensures concurrency bugs are limited to a single
device driver (i.e. a device driver can send the wrong command to a flash
storage device but cannot send a command to a radio module sharing the same SPI
bus). In particular, \name uses two features of Rust to sandbox device drivers.

\begin{enumerate}
  \item Devices drivers are compiled with a safe subset of the Rust language
  (enforced by a compiler flag), preventing device drivers from maliciously, or
  inadvertedly subverting the type or module system.

  \item Underlying hardware resources such as SPI controllers, timers, etc, are
    modeled as typed values in Rust. \name uses Rust's move semantics to ensure
    no two device drivers have a reference to the same underlying hardware
    resource. Virtualized resources (e.g. a SPI master controlling multiple SPI
    slaves) must be modeled explicitly, with a separate value for each virtual
    resource. This prevents a class of bugs that could result from concurrent
    access to hardware resources by different device drivers.

\end{enumerate} 

