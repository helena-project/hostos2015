

Microcontroller operating systems and frameworks typically assume that a single,
monolithic application will run on an embedded system.
Traditionally, the need to optimize for power and to squeeze applications into
minimal code and memory allocations have constrained microcontroller
applications to be single-function.
The absence of hardware primitives capable of providing protection has
eliminated security and isolation as considerations.
Newer hardware, however, has changed this paradigm.
The microcontroller is growing up and can now support a secure, trusted kernel
and multiple, isolated, concurrent, and dynamically-loaded applications, all
while operating on the power budgets that originally made this device class
feasible.
While the hardware support is now present, the software ecosystem to
capitalize on these advances is lagging behind.
To remedy this, we propose Tock, a new embedded operating system that builds
on established operating system principles adapted to the embedded system
environment.
Tock exploits memory protection units, advancements in modern systems
programming languages, and the event driven nature of embedded applications to
allow a core kernel, device specific drivers, and untrusted applications to
coexist on a single microcontroller.
This new operating system will allow embedded devices to mature beyond
program-once, deploy-once systems into re-usable, ubiquitous, and reliable
computing platforms.




% This is no
% longer the case with modern hardware. Modern microcontrollers, like the Atmel
% SAM4L (an ARM Cortex-M4), provides over six times more SRAM and over five times
% larger flash than the TI~MSP430 that powered the TelosB motes while maintaining
% similar power draws (90~{\uA}/MHz active and 3~\uA at sleep).
% Simultaneously embedded products are becoming a development platforms and an
% application ecosystems for products like the Pebble watch. Additionally, a set
% of modular embedded devices is emerging. Devices like SimBand, Wzzard and
% ThinkingThings have a core part to which additional modules can be attached to
% extend the functionality. These modules can be developed by the third parties
% and have own microcontroller with runtime environment.
% The operating systems community should leverage advancements in hardware,
% programming languages as well as our experience from the Web, and other
% application rich ecosystems, to build the next generation of embedded operating
% systems.

